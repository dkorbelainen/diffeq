%\documentclass[11pt,draft]{article}
\documentclass[11pt]{article}
\usepackage[T2A]{fontenc}
\usepackage[utf8]{inputenc}
\usepackage{cmap}
%\usepackage[russian, english]{babel}
\usepackage[english,russian]{babel}
\usepackage{amsmath}
\usepackage{amssymb}
\usepackage{MnSymbol}
\usepackage[active]{srcltx}
\usepackage[final]{pdfpages}
\usepackage{graphicx}
\usepackage{wrapfig}
\usepackage{tikz}
\usetikzlibrary{arrows,decorations.pathmorphing,backgrounds,positioning,fit,petri}
%\usepgflibrary{arrows} % LATEX and plain TEX and pure pgf
\usetikzlibrary{arrows} % LATEX and plain TEX when using Tik Z
\definecolor{whitesmoke}{rgb}{0.93,0.93,0.93} \definecolor{whitesmoke2}{rgb}{0.82,0.82,0.82}
\mathsurround=2pt
\setlength{\textwidth}{130mm}
\setlength{\textheight}{192mm}
%\textheight=202mm \textwidth =130mm 
\hoffset=-5mm \voffset=-20mm

\def\e{\varepsilon}
\newcommand{\pp}{\varphi}

\begin{document}

\medskip
%\noindent
Рассмотрим ЛНС с матрицей

\smallskip %\noindent
$A=\begin{pmatrix} 4&2&-2\\ -27&-9&11\\ 0&1&-1 \end{pmatrix},$ имеющей собственные числа $\lambda_{1,2,3}=-2,$ и %неоднородностью

\smallskip %\noindent
$q(x) =  \begin{pmatrix}
\alpha_0 x^{-2}e^{-2x} + \alpha_1 x^2 + \alpha_2 x + \alpha_3 + (\alpha_4 x + \alpha_5)e^{-2x} + (\alpha_6 x + \alpha_7) \sin x \\
\beta_0  x^{-2}e^{-2x} + \beta_1 x^2 + \beta_2 x  + \beta_3 +   (\beta_4 x + \beta_5)e^{-2x}   + (\beta_6 x + \beta_7) \sin x   \\
\gamma_0 x^{-2}e^{-2x} + \gamma_1 x^2 + \gamma_2 x + \gamma_3 + (\gamma_4 x + \gamma_5)e^{-2x} + (\gamma_6 x + \gamma_7) \sin x
\end{pmatrix};$

\smallskip %\noindent
Варианты неоднородностей:

1)\ $q(x) = \begin{pmatrix}
2\big(-x^{2} + x - 2 - e^{-2x}(8x - 1 - x^{-2}) + 2(12x+1)\sin x\big) \\
9(x^2 - x + 3) + e^{-2x}(56x + 1 - 3x^{-2}) - 8(21x+4)\sin x \\
-x^2 + x + 1 - e^{-2x}(8x - 1 - 3x^{-2}) + 39x\sin x
\end{pmatrix}, $

2)\ $q(x) = \begin{pmatrix}
4x^{2} - 2x + 4 - e^{-2x}(6x + 1) + (10x+6)\sin x \\
-27x^2 - 27 + e^{-2x}(27x + 9 + 2x^{-2}) - 39x \sin x\\
e^{-2x}x^{-2} + (13 x - 1) \sin x
\end{pmatrix}.$

\smallskip
{\small
1. Найдем фундаментальную матрицу линейной однородной системы

\smallskip
$y_1'=4y_1+2y_2-2y_3,\ \ \ y_2'=-27y_1-9y_2+11y_3,\ \ \ y_3'=y_2-y_3.$

\smallskip
Три  линейно независимых решения ЛОС будем искать в виде

\smallskip
$y_1=e^{-2x}(a_2x^2+a_1x+a_0),\ y_2=e^{-2x}(b_2x^2+b_1x+b_0),\ y_3=e^{-2x}(c_2x^2+c_1x+c_0).$

\smallskip
Подставим эти функции в систему, получая тождества:
$$\begin{matrix}
-2(a_2x^2+a_1x+a_0)+(2a_2x+a_1)\equiv\hfill\\
\hfill \equiv 4(a_2x^2+a_1x+a_0)+2(b_2x^2+b_1x+b_0)-2(c_2x^2+c_1x+c_0),\\
-2(b_2x^2+b_1x+b_0)+(2b_2x+b_1)\equiv\hfill\\
\hfill \equiv -27(a_2x^2+a_1x+a_0)-9(b_2x^2+b_1x+b_0)+11(c_2x^2+c_1x+c_0),\\
-2(c_2x^2+c_1x+c_0)+(2c_2x+c_1)\equiv\ (b_2x^2+b_1x+b_0)-(c_2x^2+c_1x+c_0).
\end{matrix}$$

Приравняем коэффициенты при $x^2,x^1,x^0:$

\smallskip
$\begin{cases}-2a_2=4a_2+2b_2-2c_2,\\ -2b_2=-27a_2-9b_2+11c_2,\\ -2c_2=b_2-c_2;\end{cases}\quad
\begin{cases}2a_2-2a_1=4a_1+2b_1-2c_1,\\ 2b_2-2b_1=-27a_1-9b_1+11c_1,\\ 2c_2-2c_1=b_1-c_1;\end{cases}$

$\begin{cases}a_1-2a_0=4a_0+2b_0-2c_0,\\ b_1-2b_0=-27a_0-9b_0+11c_0,\\ c_1-2c_0=b_0-c_0.\end{cases}$

\smallskip
Приведем подобные члены:

\smallskip
$\begin{cases}6a_2+2b_2-2c_2=0,\\ 27a_2+7b_2-11c_2=0,\\ b_2+c_2=0;\end{cases}\
\begin{cases}3a_1+b_1-c_1=a_2,\\ 27a_1+7b_1-11c_1=-2b_2,\\ b_1+c_1=2c_2;\end{cases}\
\begin{cases}6a_0+2b_0-2c_0=a_1,\\ 27a_0+7b_0-11c_0=-b_1,\\ b_0+c_0=c_1.\end{cases}$

\smallskip
В полученных системах выразим коэффициенты $a_j,\,b_j$ через $c_j\ \ (j=1,2,3):$

\smallskip
$ %\[
\begin{pmatrix}
a_2 \\ b_2
\end{pmatrix}
=
\begin{pmatrix}
2c_2/3 \\ -c_2
\end{pmatrix},
\quad
\begin{pmatrix}
a_1 \\ b_1
\end{pmatrix}
=
\begin{pmatrix}
2c_1/3 - 4c_2/9 \\ 2c_2 - c_1
\end{pmatrix},
\quad
\begin{pmatrix}
a_0 \\ b_0
\end{pmatrix}
=
\begin{pmatrix}
2c_0/3 - 2c_1/9 + 2c_2/27 \\ c_1 - c_0
\end{pmatrix}.
$ %\]

\smallskip
Имеет место невырожденный случай\;I, при котором три линейно независимых частных решения ЛОС ищем,
последовательно приравнивая строку коэффициентов $(c_0,c_1,c_2)$ к строкам единичной (или диагональной, чтобы убрать числовые знаменатели) матрицы и находя по ним остальные коэффициенты:
%\begin{align*}
%&c_0=3, c_1=0, c_2=0 &&\Rightarrow a_0=2, b_0=-3, a_1=0, b_1=0, a_2=0, b_2=0; \\
%&c_0=0, c_1=9, c_2=0 &&\Rightarrow a_0=-2, b_0=9, a_1=6, b_1=-9, a_2=0, b_2=0; \\
%&c_0=0, c_1=0, c_2=27 &&\Rightarrow a_0=-2, b_0=0, a_1=-12, b_1=54, a_2=18, b_2=-27.
%\end{align*}

\smallskip
$c_0=3, c_1=0, c_2=0,$ тогда \,$a_0=2, b_0=-3, a_1=0, b_1=0, a_2=0, b_2=0;$

$c_0=0, c_1=9, c_2=0,$ тогда \,$a_0=-2, b_0=9, a_1=6, b_1=-9, a_2=0, b_2=0;$

$c_0=0, c_1=0, c_2=27,$ тогда \,$a_0=2, b_0=0, a_1=-12, b_1=54, a_2=18, b_2=-27.$

\smallskip
В результате вещественная фундаментальная матрица имеет вид

\smallskip
$$\Phi(x)=e^{-2x}\begin{pmatrix} 2 & 6x - 2 & 18x^2 - 12x - 2 \\ -3 & - 9x + 9 & - 27x^2 + 54x \\ 3 & 9x & 27x^2 \end{pmatrix},$$ %\\

%\smallskip
а общее решение \,$y=\Phi(x)c$\, ЛОС, где \,$c={\rm colon}\,(c_1,c_2,c_3),$ имеет вид
\[
y=e^{-2x}\begin{pmatrix}
2\big(c_1 + c_2 (3x - 1) + c_3 (9x^2 - 6x - 1)\big) \\
3\big( -c_1-3c_2 (x-1) - 9c_3 (x^2 - 2x)\big) \\
3(c_1 + 3c_2x + 9c_3x^2)
\end{pmatrix}.
\]
%-----------------------------------------------------------------------------------------

%\smallskip
2. Неоднородность %$q(x)$
запишем в виде:
$\displaystyle q=\sum_{j=0}^3 q^{(j)}(x),$ где $q^{(0)}=x^{-2}e^{-2x}\begin{pmatrix}\alpha_0\\ \beta_0\\ \gamma_0\end{pmatrix},$
%
%\smallskip
%$q^{(0)}=x^{-2}e^{-2x}\begin{pmatrix}\alpha_0\\ \beta_0\\ \gamma_0\end{pmatrix},$
$q^{(1)}=\begin{pmatrix} \alpha_1 x^2+\alpha_2 x+\alpha_3\\ \beta_1 x^2+\beta_2 x+\beta_3\\ \gamma_1 x^2+\gamma_2 x+\gamma_3 \end{pmatrix},\ \
q^{(2)}=e^{-2x} \begin{pmatrix} \alpha_4 x + \alpha_5 \\ \beta_4 x + \beta_5 \\ \gamma_4 x + \gamma_5 \end{pmatrix},\ \
q^{(3)}=\begin{pmatrix} (\alpha_6 x + \alpha_7) \sin x \\ (\beta_6 x + \beta_7) \sin x \\ (\gamma_6 x + \gamma_7) \sin x \end{pmatrix}.$

\smallskip
$2^0.$ Частное решение для неоднородности $q^{0} = x^{-2} e^{-2x} {\rm colon}\,(\alpha_0,\beta_0,\gamma_0)$
будем искать методом вариации произвольной постоянной, считая константы из формулы общего решения ЛОС функциями $x$,
т.\,е. в виде $y=\psi^{(0)}(x),$ где \[
\begin{cases}
\psi_1^{(0)} = e^{-2x} \left( 2c_1(x) + c_2(x)(6x - 2) + c_3(x)(18x^2 - 12x - 2) \right), \\
\psi_2^{(0)} = e^{-2x} \left( -3c_1(x) + c_2(x)(-9x + 9) + c_3(x)(-27x^2 + 54x) \right), \\
\psi_3^{(0)} = e^{-2x} \left( 3c_1(x) + 9c_2(x)x + 27c_3(x)x^2 \right).
\end{cases}
\]

%\smallskip
Как было установлено, вектор производных $c'(x)$ удовлетворяет системе

\smallskip
$\Phi(x)c'(x)=q_0(x),$
или
$\begin{cases}
2c_1' + (6x - 2)c_2' + (18x^2 - 12x - 2)c_3' = \alpha_0 x^{-2}, \\
-3c_1' + (-9x + 9)c_2' + (-27x^2 + 54x)c_3' = \beta_0 x^{-2}, \\
3c_1' + 9x c_2' + 27x^2 c_3' = \gamma_0 x^{-2}.
\end{cases}$

Следовательно,
$\begin{cases} c_1' = -\Bigl( 3(9\alpha_0 + 2\beta_0 -4\gamma_0) + 2(\beta_0 + \gamma_0) x^{-1} - 2\gamma_0 x^{-2} \Bigl)/6,\\
c_2' = \Bigl(3(9\alpha_0 + 2\beta_0 - 4\gamma_0)x^{-1} + (\beta_0 + \gamma_0)x^{-2}\Bigl)/9,\\
c_3' = -(9\alpha_0 + 2\beta_0 - 4\gamma_0)x^{-2}/18.\end{cases}$

Поэтому
$\begin{cases}
c_1(x) = -\Bigl(3x (9\alpha_0 + 2\beta_0 - 4\gamma_0) + 2(\beta_0 + \gamma_0) \ln|x| + 2\gamma_0 x^{-1} \Bigl)/6,\\
c_2(x) = \Bigl(3(9\alpha_0 + 2\beta_0 - 4\gamma_0)\ln |x| - (\beta_0 + \gamma_0)x^{-1}\Bigl)/9,\\
c_3(x) = (9\alpha_0 + 2\beta_0 - 4\gamma_0)x^{-1}/18.\end{cases}$

Введем два обозначения: $K=9\alpha_0+2\beta_0-4\gamma_0$ и $L = \beta_0+\gamma_0.$
В результате
\[
\begin{cases}
\begin{aligned}
\psi_1^{(0)} &= e^{-2x} \Bigl[ 2\bigl( Kx-(K+L)/3 \bigl)\ln |x| - 2Lx/3 - 2K/3 - \alpha_0 x^{-1} \Bigr], \\
\psi_2^{(0)} &= e^{-2x} \Bigl[ \bigl( -3Kx + 3K + L \bigr)\ln |x| + Lx + 3K - \beta_0 x^{-1} \Bigr], \\
\psi_3^{(0)} &= e^{-2x} \Bigl[ \bigl( 3Kx - L \bigr)\ln |x| - Lx - \gamma_0x^{-1} \Bigr].
\end{aligned}
\end{cases}
\]

Варианты выбора параметров неоднородности:

\smallskip
1) $\begin{matrix} \alpha_0 = 2,\hfill\\ \beta_0 = -3,\\ \gamma_0 = 3,\hfill \end{matrix}$ \
тогда \ \ $\begin{cases}
\psi_1^{(0)}= -2e^{-2x} x^{-1},\\ \psi_2^{(0)}= 3e^{-2x} x^{-1},\\ \psi_3^{(0)}= -3e^{-2x} x^{-1}; \end{cases}$ \
$(K=L=0).$

\smallskip
2) $\begin{matrix} \alpha_0 = 0,\hfill\\ \beta_0 = 2,\\ \gamma_0 = 1,\hfill \end{matrix}$ \
тогда \ \ $\begin{cases}
\psi_1^{(0)} = -2e^{-2x} \big(\ln |x| + x\big), \\
\psi_2^{(0)} = e^{-2x} \big(3\ln |x| + 3x - 2x^{-1} \big), \\
\psi_3^{(0)} = -e^{-2x} \big(3\ln |x| + 3x + x^{-1} \big);
\end{cases}$ \
$(K=0, \hspace{0.5em} L=3).$

%\smallskip
%3) $\begin{matrix} \alpha_0 = 4,\hfill\\ \beta_0 = 0,\\ \gamma_0 = 9,\hfill \end{matrix}$ \
%тогда \ \ $\begin{cases}
%\psi_1^{(0)} = -2e^{-2x} \big(3\ln |x| + 2x^{-1} + 3\big), \\
%\psi_2^{(0)} = 9e^{-2x} \big(\ln |x| + 1\big), \\
%\psi_3^{(0)} = -9e^{-2x} \big(\ln |x| + x^{-1} + 1\big);
%\end{cases}$

%----------------------------------------------------------------------------------------
\smallskip
$2^1.$ Частное решение для неоднородности $q^{(1)}(x) = \begin{pmatrix} \alpha_1 x^2 + \alpha_2 x + \alpha_3 \\ \beta_1 x^2 + \beta_2 x + \beta_3 \\ \gamma_1 x^2 + \gamma_2 x + \gamma_3 \end{pmatrix}$ будем искать методом неопределенных коэффициентов.

Характеристическое число $\lambda_0=0$ неоднородности $q^{(1)}(x)$ не совпадает с собственными числами матрицы $A$,
поэтому решение ищем в виде $y=\psi^{(1)}(x),$ где

\smallskip
$\psi_1^{(1)} = a_2 x^2 + a_1 x + a_0, \quad \psi_2^{(1)} = b_2 x^2 + b_1 x + b_0, \quad \psi_3^{(1)} = c_2 x^2 + c_1 x + c_0.$

\smallskip
Подставляя эти функции в ЛНС $y'=Ay+q^{(1)}(x),$ получаем тождества
\begin{align*}
2a_2 x + a_1 &\equiv 4(a_2 x^2 + a_1 x + a_0) + 2(b_2 x^2 + b_1 x + b_0) - 2(c_2 x^2 + c_1 x + c_0) +\\
             &\quad + (\alpha_1 x^2 + \alpha_2 x + \alpha_3), \\
2b_2 x + b_1 &\equiv -27(a_2 x^2 + a_1 x + a_0) - 9(b_2 x^2 + b_1 x + b_0) + 11(c_2 x^2 + c_1 x + c_0) +\\
             &\quad + (\beta_1 x^2 + \beta_2 x + \beta_3), \\
2c_2 x + c_1 &\equiv (b_2 x^2 + b_1 x + b_0) - (c_2 x^2 + c_1 x + c_0) + (\gamma_1 x^2 + \gamma_2 x + \gamma_3).
\end{align*}

\smallskip
Приравняем коэффициенты при линейно независимых функциях:
%Приравнивая коэффициенты при $x^2,x^1,x^0,$ имеем:
$$x^2:
\begin{cases}
4a_2 + 2b_2 - 2c_2 + \alpha_1 = 0, \\
27a_2 + 9b_2 - 11c_2 - \beta_1 = 0, \\
b_2 - c_2 + \gamma_1 = 0;
\end{cases} \quad
x^1:
\begin{cases}
2a_2 - 4a_1 - 2b_1 + 2c_1 - \alpha_2 = 0, \\
2b_2 + 27a_1 + 9b_1 - 11c_1 - \beta_2 = 0, \\
2c_2 -b_1 + c_1 - \gamma_2 = 0;
\end{cases}
$$
$$x^0:
\begin{cases}
a_1 - 4a_0 - 2b_0 + 2c_0 - \alpha_3 = 0, \\
b_1 + 27a_0 + 9b_0 - 11c_0 - \beta_3 = 0, \\
c_1 - b_0 + c_0 - \gamma_3 = 0.
\end{cases}
$$

Решая системы последовательно, начиная с последней, находим значения коэффициентов:
\[
\begin{array}{ll}
a_2 = (-\alpha_1 + 2\gamma_1)/4, b_2 = (-27\alpha_1 - 4\beta_1 + 10\gamma_1)/8, c_2 = (-27\alpha_1 - 4\beta_1 + 18\gamma_1)/8, & \\[6pt]
a_1 = (13\alpha_1 + 2\beta_1 - 8\gamma_1  - \alpha_2 + 2\gamma_2)/4, b_1 = (27\alpha_1 + 6\beta_1 - 8\gamma_1 -27\alpha_2 - 4\beta_2 + 10\gamma_2)/8, & \\
c_1 = (81\alpha_1 + 14\beta_1 - 44\gamma_1 - 27\alpha_2 - 4\beta_2 + 18\gamma_2)/8, & \\[6pt]
a_0 = (-34\alpha_1 - 6\beta_1 + 18\gamma_1 + 13\alpha_2 + 2\beta_2 - 8\gamma_2 - 2\alpha_3 + 4\gamma_3)/8, & \\[3pt]
b_0 = (-2\beta_1 - 6\gamma_1 + 27\alpha_2 + 6\beta_2 - 8\gamma2/ - 54\alpha_3 - 8\beta_3 + 20\gamma_3)/16, & \\
c_0 = (-162\alpha_1 - 30\beta_1 + 82\gamma_1 + 81\alpha_2 + 14\beta_2 - 44\gamma_2 - 54\alpha_3 -8\beta_3 +36\gamma_3)/16. &
\end{array}
\]

Следовательно,
\[
\left\{
\begin{aligned}
\psi_1^{(1)} &= \big[2(- \alpha_1 + 2\gamma_1)x^2 + 2(13\alpha_1 + 2\beta_1 -8\gamma_1 - \alpha_2 + 2\gamma_2)x - \\
&\quad -34\alpha_1 - 6\beta_1 + 18\gamma_1 + 13\alpha_2 + 2\beta_2 - 8\gamma_2 - 2\alpha_3 + 4\gamma_3\big]/8, \\
\psi_2^{(1)} &= \big[2( - 27\alpha_1 - 4\beta_1 + 10\gamma_1)x^2 + 2(27\alpha_1 + 6\beta_1 - 8\gamma_1 - 27\alpha_2 - 4\beta_2 + 10\gamma_2)x - \\
&\quad - 2\beta_1 - 6\gamma_1 + 27\alpha_2 + 6\beta_2 - 8\gamma_2 - 54\alpha_3 - 8\beta_3 + 20\gamma_3\big]/16, \\
\psi_3^{(1)} &= \big[2( - 27\alpha_1 - 4\beta_1 + 18\gamma_1)x^2 + 2(81\alpha_1 + 14\beta_1 - 44\gamma_1 -27\alpha_2 - 4\beta_2 + 18\gamma_2)x - \\
&\quad -162\alpha_1 - 30\beta_1 + 82\gamma_1 + 81\alpha_2 + 14\beta_2 - 44\gamma_2 -54\alpha_3 - 8\beta_3 + 36\gamma_3\big]/16.
\end{aligned}
\right.
\]

Варианты выбора коэффициентов:

\smallskip
1) $\begin{matrix} \alpha_1 = -2,\hfill\\ \beta_1 = 9,\\ \gamma_1 = -1,\hfill \end{matrix}$ \
$\begin{matrix} \alpha_2 = 2,\hfill\\ \beta_2 = -9,\\ \gamma_2 = 1,\hfill \end{matrix}$ \
$\begin{matrix} \alpha_3 = -4,\hfill\\ \beta_3 = 27,\\ \gamma_3 = 1,\hfill \end{matrix}$
тогда \ \ $\begin{cases}
\psi_1^{(1)} = 1, \\
\psi_2^{(1)} = x^2, \\
\psi_3^{(1)} = x;
\end{cases}$

\smallskip
2) $\begin{matrix} \alpha_1 = 4,\hfill\\ \beta_1 = -27,\\ \gamma_1 = 0,\hfill \end{matrix}$ \
$\begin{matrix} \alpha_2 = -2,\hfill\\ \beta_2 = 0,\\ \gamma_2 = 0,\hfill \end{matrix}$ \
$\begin{matrix} \alpha_3 = 4,\hfill\\ \beta_3 = -27,\\ \gamma_3 = 0,\hfill \end{matrix}$
тогда \ \ $\begin{cases}
\psi_1^{(1)} = -x^2 -1, \\
\psi_2^{(1)} = 0, \\
\psi_3^{(1)} = 0;
\end{cases}$

%\smallskip
%3) $\begin{matrix} \alpha_1 = 2,\hfill\\ \beta_1 = -11,\\ \gamma_1 = 1,\hfill \end{matrix}$ \
%$\begin{matrix} \alpha_2 = 0,\hfill\\ \beta_2 = 0,\\ \gamma_2 = 2,\hfill \end{matrix}$ \
%$\begin{matrix} \alpha_3 = -6,\hfill\\ \beta_3 = 36,\\ \gamma_3 = -1,\hfill \end{matrix}$
%тогда \ \ $\begin{cases}
%\psi_1^{(1)} = 1, \\
%\psi_2^{(1)} = 1, \\
%\psi_3^{(1)} = x^2.
%\end{cases}$

%-----------------------------------------------------------------------------------------------

\smallskip
$2^2.$ Частное решение для неоднородности $q^{(2)}(x) = e^{-2x} \begin{pmatrix} \alpha_4 x + \alpha_5 \\ \beta_4 x + \beta_5 \\ \gamma_4 x + \gamma_5 \end{pmatrix}$ ищем методом неопределённых коэффициентов.

Характеристическое число $\lambda_0 = -2$ неоднородности $q^{(2)}(x)$ совпадает с собственным числом матрицы $A$, причем $m=1$, $\tilde{k} = k = s = 3$, поэтому решение ищем в виде $y=\psi^{(2)}(x),$ где

\[
\begin{cases}
\psi_1^{(2)} = e^{-2x} (a_4 x^4 + a_3 x^3 + a_2 x^2 + a_1 x + a_0), \\
\psi_2^{(2)} = e^{-2x} (b_4 x^4 + b_3 x^3 + b_2 x^2 + b_1 x + b_0), \\
\psi_3^{(2)} = e^{-2x} (c_4 x^4 + c_3 x^3 + c_2 x^2 + c_1 x + c_0).
\end{cases}
\]

Выбираем $c_2 = c_1 = c_0 = 0$ и подставляем эти функции в ЛНС $y'=Ay+q^{(2)}(x),$ получая тождества:

\[
\begin{aligned}
&-2(a_4x^4 + a_3x^3 + a_2x^2 + a_1x + a_0) + (4a_4x^3 + 3a_3x^2 + 2a_2x + a_1) \equiv \\
&\quad \equiv4(a_4x^4 + a_3x^3 + a_2x^2 + a_1x + a_0) + 2(b_4x^4 + b_3x^3 + b_2x^2 + b_1x + b_0)- \\
&\quad - 2(c_4x^4 + c_3x^3) + (\alpha_4x + \alpha_5), \\
%
&-2(b_4x^4 + b_3x^3 + b_2x^2 + b_1x + b_0) + (4b_4x^3 + 3b_3x^2 + 2b_2x + b_1) \equiv \\
&\quad \equiv-27(a_4x^4 + a_3x^3 + a_2x^2 + a_1x + a_0) - 9(b_4x^4 + b_3x^3 + b_2x^2 + b_1x + b_0)+ \\
&\quad + 11(c_4x^4 + c_3x^3) + (\beta_4x + \beta_5), \\
%
&-2(c_4x^4 + c_3x^3) + (4c_4x^3 + 3c_3x^2) \equiv \\
&\quad \equiv(b_4x^4 + b_3x^3 + b_2x^2 + b_1x + b_0) - (c_4x^4 + c_3x^3) + (\gamma_4x + \gamma_5).
\end{aligned}
\]

Приравниваем коэффициенты при линейно независимых функциях:
\[
\begin{aligned}
x^4: & \begin{cases}
-6a_4 - 2b_4 + 2c_4 = 0, \\
27a_4 + 7b_4 - 11c_4 = 0, \\
-b_4 - c_4 = 0;
\end{cases} \hspace{0.25em}
x^3: \begin{cases}
-6a_3 + 4a_4 - 2b_3 + 2c_3 = 0, \\
27a_3 + 7b_3 + 4b_4 - 11c_3 = 0, \\
-b_3 - c_3 + 4c_4 = 0;
\end{cases} \\
x^2: & \begin{cases}
-6a_2 + 3a_3 - 2b_2 = 0, \\
27a_2 + 7b_2 + 3b_3 = 0, \\
-b_2 + 3c_3 = 0;
\end{cases}
x^1: \begin{cases}
-6a_1 + 2a_2 - 2b_1 = \alpha_4, \\
27a_1 + 7b_1 + 2b_2 = \beta_4, \\
-b_1 = \gamma_4;
\end{cases}
x^0: \begin{cases}
-6a_0 + a_1 - 2b_0 = \alpha_5, \\
27a_0 + 7b_0 + b_1 = \beta_5, \\
-b_0 = \gamma_5.
\end{cases}
\end{aligned}
\]

\begin{align*}
a_0 &= (\beta_5 + 7\gamma_5 + \gamma_4)/27, \quad a_1 = \alpha_5 + 2(\beta_5 - 2\gamma_5 + \gamma_4)/9, \\a_2 &= (3\alpha_4 + 18\alpha_5 + 4\beta_5 - 8\gamma_5 - 2\gamma_4)/6, \\
a_3 &= \alpha_4 - 3\alpha_5 + (\beta_4 - 2\beta_5 + 4\gamma_5 - \gamma_4)/3, \quad a_4 = (-9\alpha_4-2\beta_4+4\gamma_4)/8, \\[3pt]
b_0 &= -\gamma_5, \quad b_1 = -\gamma_4, \quad b_2 = - 3\beta_5 + 6\gamma_5 + (\gamma_4 + \beta_4 - 27\alpha_5)/2,\\
b_3 &= \beta_5 - 2\gamma_5 -(27\alpha_4 - 27\alpha_5 + 7\beta_4 - 11\gamma_4)/6, \quad b_4 = (9\alpha_4 + 2\beta_4 - 4\gamma_4)/8, \\[3pt]
c_3 &= 2\gamma_5 - \beta_5 -(27\alpha_5 - \beta_4 - \gamma_4)/6, \quad c_4 = (-9\alpha_4-2\beta_4+4\gamma_4)/8.
\end{align*}

Следовательно,
\begin{equation*}
\begin{cases}
\psi_1^{(2)} = e^{-2x} \Bigl[ 
6(3\alpha_4 + \beta_4 - \gamma_4 - 9 \alpha_5 - 2\beta_5 + 4\gamma_5) x^3 + \\
\quad + 3( 3\alpha_4 - 2\gamma_4 + 18 \alpha_5 + 4\beta_5 - 8\gamma_5) x^2 + \\
\quad + 2( 2\gamma_4 + 9\alpha_5 + 2\beta_5 - 4\gamma_5 ) x + 2(\gamma_4 + \beta_5 + 7\gamma_5) \Bigl]/18, \\
\psi_2^{(2)} = e^{-2x} \Bigl[ 
(-27\alpha_4 -7\beta_4 + 11\gamma_4 + 27\alpha_5 + 6\beta_5 - 12\gamma_5) x^3 + \\
\quad + 3( \beta_4 + \gamma_4 - 27\alpha_5 -6\beta_5 + 12\gamma_5 ) x^2 - 6\gamma_4 x - 6\gamma_5 \Bigl]/6, \\
\psi_3^{(2)} = e^{-2x} \Bigl[ 
\beta_4 + \gamma_4 - 27\alpha_5 - 6\beta_5 + 12\gamma_5 \Bigl] x^3 /6.
\end{cases}
\end{equation*}
И здесь степень многочлена <<случайно>> оказалась ниже четвертой.

Варианты выбора коэффициентов:

\smallskip
1) $\begin{matrix} \alpha_4 = -16,\hfill\\ \beta_4 = 56,\\ \gamma_4 = -8,\hfill \end{matrix}$
$\begin{matrix} \alpha_5 = 2,\hfill\\ \beta_5 = 1,\\ \gamma_5 = 1,\hfill \end{matrix}$ \ 
тогда \ \ $\begin{cases} 
\psi_1^{(2)} = 0, \\
\psi_2^{(2)} = e^{-2x} (8x - 1), \\
\psi_3^{(2)} = 0;
\end{cases}$

\smallskip

2) $\begin{matrix} \alpha_4 = -6,\hfill\\ \beta_4 = 27,\\ \gamma_4 = 0,\hfill \end{matrix}$
$\begin{matrix} \alpha_5 = -1,\hfill\\ \beta_5 = 9,\\ \gamma_5 = 0,\hfill \end{matrix}$ \ 
тогда \ \ $\begin{cases} 
\psi_1^{(2)} = e^{-2x} (x+1), \\
\psi_2^{(2)} = 0, \\
\psi_3^{(2)} = 0.
\end{cases}$ 

%\smallskip
%3) $\begin{matrix} \alpha_4 = -4,\hfill\\ \beta_4 = 20,\\ \gamma_4 = 1,\hfill \end{matrix}$
%$\begin{matrix} \alpha_5 = 1,\hfill\\ \beta_5 = 1,\\ \gamma_5 = 1,\hfill \end{matrix}$ \ 
%тогда \ \ $\begin{cases} 
%\psi_1^{(2)} = e^{-2x} (x+1), \\
%\psi_2^{(2)} = -e^{-2x} (x+1), \\
%\psi_3^{(2)} = 0.
%\end{cases}$

%------------------------------------------------------------------------------------------

\smallskip
$2^3.$ Частное решение для $q_3(x) = \begin{pmatrix} (\alpha_6 x + \alpha_7) \sin x \\ (\beta_6 x + \beta_7) \sin x \\ (\gamma_6 x + \gamma_7) \sin x \end{pmatrix}$ ищем методом неопределенных коэффициентов.

Характеристическое число $\lambda_0=$ $i$ неоднородности $q^{(3)}(x)$ не совпадает с собственными числами матрицы $A$, поэтому решение ищем в виде \\ $\psi_j^{(3)} = (a_j x + b_j) \sin x + (c_j x + d_j) \cos x \quad (j=1,2,3).$

Подставляя эти функции в ЛНС $y'=Ay+q^{(3)}(x),$ получаем тождества
\begin{align*}
&(a_1 \sin x + c_1 \cos x + (a_1 x + b_1)\cos x - (c_1 x + d_1)\sin x)\equiv \\
&\equiv \big(4(a_1 x + b_1)\sin x + 2(a_2 x + b_2)\sin x - 2(a_3 x + b_3)\sin x+ \\
&\quad + (\alpha_6 x + \alpha_7)\sin x + 4(c_1 x + d_1)\cos x + 2(c_2 x + d_2)\cos x - 2(c_3 x + d_3)\cos x\big),
\end{align*}
\vspace{-2em}
\begin{align*}
&(a_2 \sin x + c_2 \cos x + (a_2 x + b_2)\cos x - (c_2 x + d_2)\sin x)\equiv \\
&\equiv \big(-27(a_1 x + b_1)\sin x - 9(a_2 x + b_2)\sin x + 11(a_3 x + b_3)\sin x+ \\
&\quad + (\beta_6 x + \beta_7)\sin x - 27(c_1 x + d_1)\cos x - 9(c_2 x + d_2)\cos x+ 11(c_3 x + d_3)\cos x\big),
\end{align*}
\vspace{-2em}
\begin{align*}
&(a_3 \sin x + c_3 \cos x + (a_3 x + b_3)\cos x - (c_3 x + d_3)\sin x)\equiv \\
&\equiv \big((a_2 x + b_2)\sin x - (a_3 x + b_3)\sin x + (c_2 x + d_2)\cos x- \\
&\quad - (c_3 x + d_3)\cos x + (\gamma_6 x + \gamma_7)\sin x\big).
\end{align*}

\smallskip
Приравняем коэффициенты при $x \cos x, \cos x, x \sin x, \sin x$:

$x \cos x: \begin{cases} 
a_1 - 4c_1 - 2c_2 + 2c_3 = 0, \\ 
a_2 + 27c_1 + 9c_2 - 11c_3 = 0, \\ 
a_3 - c_2 + c_3 = 0; 
\end{cases}
\cos x: \begin{cases} 
b_1 + c_1 - 4d_1 - 2d_2 + 2d_3 = 0, \\ 
b_2 + c_2 + 27d_1 + 9d_2 - 11d_3 = 0, \\ 
b_3 + c_3 - d_2 + d_3 = 0; 
\end{cases}$

\hspace{-1em}
$x \sin x: \begin{cases} 
-4a_1 - 2a_2 + 2a_3 - c_1 = \alpha_6, \\ 
27a_1 + 9a_2 - 11a_3 - c_2 = \beta_6, \\ 
-a_2 + a_3 - c_3 = \gamma_6; 
\end{cases}
\sin x: \begin{cases} 
a_1 - 4b_1 - 2b_2 + 2b_3 - d_1 = \alpha_7, \\ 
a_2 + 27b_1 + 9b_2 - 11b_3 - d_2 = \beta_7, \\ 
a_3 - b_2 + b_3 - d_3 = \gamma_7.
\end{cases}$

Решая системы последовательно, находим значения коэффициентов:
\begin{align*}
a_1 &= 2(52 \alpha_6 + 11 \beta_6 - 7 \gamma_6)/125, \\
b_1 &= (-573 \alpha_6 - 124 \beta_6 + 208 \gamma_6 + 520 \alpha_7 + 110 \beta_7 - 70 \gamma_7)/625, \\
c_1 &= (53 \alpha_6 + 4 \beta_6 - 48 \gamma_6)/125, \\
d_1 &= (-536 \alpha_6 - 68 \beta_6 + 356 \gamma_6 + 265 \alpha_7 + 20 \beta_7 - 240 \gamma_7)/625, \\
a_2 &= (-351 \alpha_6 - 43 \beta_6 + 141 \gamma_6)/125, \\
b_2 &= (1107 \alpha_6 + 191 \beta_6 - 472 \gamma_6 - 1755 \alpha_7 - 215 \beta_7 + 705 \gamma_7)/625, \\
c_2 &= (243 \alpha_6 + 49 \beta_6 - 88 \gamma_6)/125, \\
d_2 &= (-1026 \alpha_6 - 238 \beta_6 + 346 \gamma_6 + 1215 \alpha_7 + 245 \beta_7 - 440 \gamma_7)/625, \\
a_3 &= (-54 \alpha_6 + 3 \beta_6 + 89 \gamma_6)/125, \\
b_3 &= (-567 \alpha_6 - 146 \beta_6 + 157 \gamma_6 - 270 \alpha_7 + 15 \beta_7 + 445 \gamma_7)/625, \\
c_3 &= (297 \alpha_6 + 46 \beta_6 - 177 \gamma_6)/125, \\
d_3 &= (-1944 \alpha_6 - 322 \beta_6 + 1074 \gamma_6 + 1485 \alpha_7 + 230 \beta_7 - 885 \gamma_7)/625.
\end{align*}
Следовательно,
\begin{equation*}
\begin{cases}
\psi_1^{(3)} = \Big[ \Big( 10(52\alpha_6 + 11\beta_6 - 7\gamma_6)x - 573\alpha_6 - 124\beta_6 + 208\gamma_6 + 520\alpha_7 + 110\beta_7 - \\
\quad - 70\gamma_7 \Big) \sin x + \Big( 5(53\alpha_6 + 4\beta_6 - 48\gamma_6)x - 536\alpha_6 - 68\beta_6 + 356\gamma_6 + 265\alpha_7 + 20\beta_7 -  \\
\quad - 240\gamma_7 \Big) \cos x \Big]/625, \\
\psi_2^{(3)} = \Big[ \Big(- 5(351\alpha_6 + 43\beta_6 - 141\gamma_6)x + 1107\alpha_6 + 191\beta_6 - 472\gamma_6 - 1755\alpha_7 - \\ 
\quad - 215\beta_7 + 705\gamma_7 \Big) \sin x + \Big(5x(243\alpha_6 + 49\beta_6 - 88\gamma_6) - 1026\alpha_6 - 238\beta_6 + \\ 
\quad + 346\gamma_6 + 1215\alpha_7 + 245\beta_7 - 440\gamma_7 \Big) \cos x \Big]/625, \\
\psi_3^{(3)} = \Big[ \Big(- 5x(54\alpha_6 - 3\beta_6 - 89\gamma_6) -567\alpha_6 - 146\beta_6 + 157\gamma_6 - 270\alpha_7 + 15\beta_7 + \\
\quad + 445\gamma_7 \Big) \sin x + \Big(5x(297\alpha_6 + 46\beta_6 - 177\gamma_6) - 1944\alpha_6 - 322\beta_6 + 1074\gamma_6 + \\ 
\quad + 1485\alpha_7 + 230\beta_7 - 885\gamma_7 \Big) \cos x \Big]/625.
\end{cases}
\end{equation*}

\smallskip
Варианты выбора коэффициентов: 

\smallskip
1) $\begin{matrix} \alpha_6 = 48,\hfill\\ \beta_6 = -168,\\ \gamma_6 = 39,\hfill \end{matrix}$
$\begin{matrix} \alpha_7 = 4,\hfill\\ \beta_7 = -32,\\ \gamma_7 = 0,\hfill \end{matrix}$ \ 
тогда \ \ $\begin{cases} 
\psi_1^{(3)} = 6x\sin x, \\
\psi_2^{(3)} = (-33x+4) \sin x + 2\cos x, \\
\psi_3^{(3)} = 3(x+1) \sin x - (3x-2) \cos x;
\end{cases}$

\smallskip
2) $\begin{matrix} \alpha_6 = 10,\hfill\\ \beta_6 = -39,\\ \gamma_6 = 13,\hfill \end{matrix}$
$\begin{matrix} \alpha_7 = 6,\hfill\\ \beta_7 = 0,\\ \gamma_7 = -1,\hfill \end{matrix}$ \ 
тогда \ \ $\begin{cases} 
\psi_1^{(3)} = 8\sin x - 2(x-3)\cos x, \\
\psi_2^{(3)} = -22\sin x - (5x-18) \cos x, \\
\psi_3^{(3)} = 4x \sin x - 9(x - 3)\cos x.
\end{cases}$

%\smallskip
%3) $\alpha_6=1250, \hspace{0.5em} \beta_6=-4875, \hspace{0.5em} \gamma_6=1625, \hspace{0.5em} \alpha_7=0, %\hspace{0.5em} \beta_7=11375, \gamma_7=17875.$
%\begin{equation*}
%\begin{array}{l}
%\text{Получим: } 
%\begin{cases}
%\begin{aligned}
%y_1 &= 362\sin x - 2\cos x (125x + 3058) \\
%y_2 &= 15747\sin x - \cos x (625x + 7421) \\
%y_3 &= \sin x (500x + 13413) - \cos x (1125x + 19709)
%\end{aligned}
%\end{cases}
%\end{array}
%\end{equation*}

\bigskip
\textbf{Ответ:}
$y=\Phi(x)c + \psi^{(0)} + \psi^{(1)} + \psi^{(2)} + \psi^{(3)}.$

\bigskip
\textbf{Ответ для первого $q(x)$:}
\begin{equation*}
\begin{cases}
\begin{aligned}
y_1 &= 2e^{-2x} \big( 9c_3 x^2 + 3(c_2 - 2c_3)x + c_1 - c_2 - c_3 - x^{-1} \big) + 1 + 6x\sin x, \\
y_2 &= -e^{-2x} \big( 27c_3 x^2 + (9c_2 - 54c_3 - 8)x + 3c_1 - 9c_2 + 1 - 3x^{-1} \big) + x^2 - \\
& \quad - (33x-4)\sin x + 2\cos x, \\
y_3 &= 3e^{-2x} \big( 9c_3 x^2 + 3c_2x + c_1 - x^{-1} \big) + x + 3(x+1)\sin x - (3x-2) \cos x .
\end{aligned}
\end{cases}
\end{equation*}

\textbf{Ответ для второго $q(x)$:}
\begin{equation*}
\begin{cases}
\begin{aligned}
y_1 &= e^{-2x} \big( 18c_3 x^2 + (6c_2 - 12c_3 - 1)x + 2c_1 - 2c_2 - 2c_3 + 1 - 2 \ln|x| \big) - x^2 - 1 + \\
& \quad + 8\sin x - 2(x-3) \cos x, \\
y_2 &= -e^{-2x} \big( 27c_3x^2 + 3(3c_2 - 18c_3 - 1)x + 3c_1 - 9c_2 - 3\ln|x| + 2x^{-1} \big) - \\
& \quad - 22\sin x - (5x - 18)\cos x, \\
y_3 &= e^{-2x} \big( 27c_3 x^2 + 3(3c_2-1) x + 3c_1 - 3\ln|x| - x^{-1} \big) + x^2 +4x\sin x - 9(x-3)\cos x.
\end{aligned}
\end{cases}
\end{equation*}

}

\end{document}